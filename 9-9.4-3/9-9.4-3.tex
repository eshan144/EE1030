\let\negmedspace\undefined
\let\negthickspace\undefined
\documentclass[journal]{IEEEtran}
\usepackage[a5paper, margin=10mm, onecolumn]{geometry}
%\usepackage{lmodern} % Ensure lmodern is loaded for pdflatex
\usepackage{tfrupee} % Include tfrupee package

\setlength{\headheight}{1cm} % Set the height of the header box
\setlength{\headsep}{0mm}     % Set the distance between the header box and the top of the text

\usepackage{xparse}
\usepackage{gvv-book}
\usepackage{gvv}
\usepackage{cite}
\usepackage{amsmath,amssymb,amsfonts,amsthm}
\usepackage{algorithmic}
\usepackage{graphicx}
\usepackage{textcomp}
\usepackage{xcolor}
\usepackage{txfonts}
\usepackage{listings}
\usepackage{enumitem}
\usepackage{mathtools}
\usepackage{gensymb}
\usepackage{comment}
\usepackage[breaklinks=true]{hyperref}
\usepackage{tkz-euclide} 
\usepackage{listings}
\usepackage{gvv}                                        
\def\inputGnumericTable{}                                 
\usepackage[latin1]{inputenc}                                
\usepackage{color}                                            
\usepackage{array}                                            
\usepackage{longtable}                                       
\usepackage{calc}                                             
\usepackage{multirow}                                         
\usepackage{hhline}                                           
\usepackage{ifthen}                                           
\usepackage{lscape}
\begin{document}

\bibliographystyle{IEEEtran}
\vspace{3cm}

\title{9-9.4-3}
\author{EE24BTECH11022 - Eshan Sharma}
% \maketitle
% \newpage
% \bigskip
{\let\newpage\relax\maketitle}

\renewcommand{\thefigure}{\theenumi}
\renewcommand{\thetable}{\theenumi}
\setlength{\intextsep}{10pt} % Space between text and floats


\numberwithin{equation}{enumi}
\numberwithin{figure}{enumi}
\renewcommand{\thetable}{\theenumi}

\textbf{Question}:\\
Find the area of the circle $x^{2}+y^{2} = 16$ exterior to the parabola $y^{2} = 6x$.
\\
\begin{table}[h!]    
  \centering
  \begin{tabular}[20pt]{ |c|c|c|c|c| }
    \hline
    \textbf{Activity} & \textbf{A} & \textbf{B} & \textbf{C} & \textbf{D} \\
    \hline
    \textbf{Mean(days)} & $6$ & $11$ & $8$ & $15$\\
    \hline 
    \textbf{Variance(days$^{2}$)} & $4$ & $9$ & $4$ & $9$\\
    \hline
\end{tabular}

  \caption{Variables Used}
  \label{tab0}
\end{table}
\solution\\
The given circle $\vec{C}$ can be expressed as
\begin{align}
	v = \myvec{1&0\\0&1}, u = 0, f = -16
\end{align}
The given parabola $\vec{P}$ can be expressed as
\begin{align}
	v = \myvec{0&0\\0&1}, u = \myvec{-3\\0}, f = 0
\end{align}
The area exterior to the parabola, $\vec{A}$ is given by
\begin{align}
	\vec{A} = A_c - A_i \text{, where}
\end{align}
\begin{align}
	A_c = \pi \text{ (radius of $\vec{C}$)}^{2}
\end{align}	
\begin{align}
	A_i = 2\brak{\int_{0}^{x\brak{\vec{I_1}}} \sqrt{6x} \, dx + \int_{x\brak{\vec{I_1}}}^{x\brak{\vec{I_2}}} \sqrt{16-x^{2}} \, dx}
\end{align}
\begin{align}
	\vec{A} = 16\pi - \frac{4}{3}\brak{4\pi + \sqrt{3}} \text{ units }
\end{align}
\begin{align}
	\vec{A} = \frac{4}{3}\brak{8\pi + \sqrt{3}} \text{ units }
\end{align}

\begin{figure}[ht]
    \centering
    \includegraphics[width=\linewidth]{figs/9-9.2-41.png}
    \caption{Area enclosed in the first quadrant}
\end{figure}
  
\end{document}
