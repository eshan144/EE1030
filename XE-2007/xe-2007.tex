%iffalse
\let\negmedspace\undefined
\let\negthickspace\undefined
\documentclass[journal,12pt,onecolumn]{IEEEtran}
\usepackage{cite}
\usepackage{amsmath,amssymb,amsfonts,amsthm}
\usepackage{algorithmic}
\usepackage{graphicx}
\usepackage{textcomp}
\usepackage{xcolor}
\usepackage{txfonts}
\usepackage{listings}
\usepackage{enumitem}
\usepackage{enumitem,multicol}
\usepackage{mathtools}
\usepackage{gensymb}
\usepackage{comment}
\usepackage[breaklinks=true]{hyperref}
\usepackage{tkz-euclide} 
\usepackage{listings}
\usepackage{gvv}                                        
%\def\inputGnumericTable{}                                 
\usepackage[latin1]{inputenc}                                
\usepackage{color}                                            
\usepackage{array}                                            
\usepackage{longtable}                                       
\usepackage{calc}                                             
\usepackage{multirow}                                         
\usepackage{hhline}                                           
\usepackage{ifthen}                                           
\usepackage{lscape}
\usepackage{tabularx}
\usepackage{array}
\usepackage{float}
\usepackage[american,siunitx]{circuitikz}
\usetikzlibrary{arrows,shapes,calc,positioning}
\usepackage{pgfplots}


\newtheorem{theorem}{Theorem}[section]
\newtheorem{problem}{Problem}
\newtheorem{proposition}{Proposition}[section]
\newtheorem{lemma}{Lemma}[section]
\newtheorem{corollary}[theorem]{Corollary}
\newtheorem{example}{Example}[section]
\newtheorem{definition}[problem]{Definition}
\newcommand{\BEQA}{\begin{eqnarray}}
\newcommand{\EEQA}{\end{eqnarray}}
\newcommand{\define}{\stackrel{\triangle}{=}}
\theoremstyle{remark}
\newtheorem{rem}{Remark}

% Marks the beginning of the document
\begin{document}
\bibliographystyle{IEEEtran}
\vspace{3cm}

\title{XE-2007}
\author{EE24Btech11022 - Eshan Sharma}
\maketitle

\renewcommand{\thefigure}{\theenumi}
\renewcommand{\thetable}{\theenumi}



\begin{enumerate}
	Consider the following C program segment
	
	\lstset{language=C, frame=none, basicstyle=\ttfamily}
	\begin{lstlisting}
	#include <stdio.h>
	void print_mat (int [][3]);
	void main(){
		int i,j,sum=0;
		int m[3][3] = {{1,3,5},{7,9,11},{13,15,17}};
		for(i=0;i<3;i++){
			for(j=2;j>1;j--){
				sum += m[i][j]*m[i][j-1];	
			}
		}
		printf("%d",sum);
		print_mat(m);  // FUNCTION CALL	
	}
		
	void print_mat(int mat[][3]){
		int(*p)[3]=&mat[1];
		printf("%d and %d", (*p)[1], (*p)[2]);
	}
	\end{lstlisting}
    \item The values printed after the function call(marked as FUNCTION CALL) are
    \hfill{(xe-2007)}
    \begin{multicols}{4}
    \begin{enumerate}
    \item $3 \text{ and } 5$
    \item $7 \text{ and } 9$
    \item $9 \text{ and } 11$
    \item $13 \text{ and } 15$
    \end{enumerate}
    \end{multicols}
    
	Consider the following quadrature formula \\
	$\int_0^1 12f(x) \, dx = \brak{f(0)+2bf(0.25)+2f(0.5)+2df(0.75)+f(1)}.$\\
    
    \item If the above formula is used as Simpson's $1/3$ rule, then
    \hfill{(xe-2007)}
    \begin{multicols}{4}
    \begin{enumerate}
    \item $b=d=1$
    \item $b=d=2$
    \item $b=2d=1$
    \item $b=2d=2$
    \end{enumerate}
    \end{multicols}
    \item Using the correct values of b and d from above part in the quadrature formula, the value of $\int_{0}^{1} \frac{12}{1+x} \, dx$ evaluated correct up to 4 decimals is
    \hfill{(xe-2007)}
    \begin{multicols}{4}
    \begin{enumerate}
    \item $8.3091$
    \item $8.3121$
    \item $8.3151$
    \item $8.3191$
    \end{enumerate}
    \end{multicols}
    
    Consider the initial value problem $\frac{dy}{dx} = f(x,y) = 2xy$ with $y(0)=1, y(0.2) = 1.0408, y(0.4) = 1.1735 \text{ and } y(0.6) = 1.4333.$ \\
    
    \item Choose the correct predictor scheme to solve the above initial problem at $x=0.8$ from the following
    \hfill{(xe-2007)} 
    \begin{enumerate}
    \item $y_{n+1} = y_n + \frac{4h}{3}\brak{2f_{n-1} - f_{n-2} + 2f_{n-3}}$
    \item $y_{n+1} = y_{n-3} + \frac{4h}{3}\brak{2f_{n-2} - f_{n-1} + 2f_n}$
    \item $y_{n+1} = y_{n-1} + \frac{h}{3}\brak{4f_{n-1} - 5f_{n} + 4f_{n+1}}$
    \item $y_{n+1} = y_{n-3} + \frac{4h}{3}\brak{2f_{n-1} - f_{n-2} + 2f_{n-3}}$\\
    \end{enumerate}
    \item Using the correct predictor scheme from above, the value of $y(0.8)$ is
    \hfill{(xe-2007)}
    \begin{enumerate}
    \begin{multicols}{4}
    \item $1.8680$
    \item $1.8750$
    \item $1.8890$
    \item $1.9055$
	\end{multicols}
    \end{enumerate}
    \item Assuming all components are ideal, the average power delivered by the dc voltage source network shown in the figure is
    \hfill{(xe-2007)}
    \begin{center}
    	\begin{circuitikz} 
    		\draw
    		(0,0) to[isource, l_= $10\cos(100\pi t) \text{A} $] (0,4)
    		to[R=$1\, \Omega$] (4,4) -- (4,0)
    		to[battery, l_=$8 \text{V}$] (0,0)
    		(0,0) -- (0,-2)
    		to[D, invert] (4,-2) -- (4,0);
    	\end{circuitikz}
    \end{center}
    
    \begin{multicols}{4} 
    \begin{enumerate}
    \item $-28$ W
    \item $0$ W
    \item $64$ W
    \item $80$ W
    \end{enumerate}
    \end{multicols}
    \item An ideal transformer with 10 turns in primary and 30 turns in secondary has its primary connected to external circuits as shown in the figure.
    \hfill{(xe-2007)}
    \begin{center}
    	\begin{circuitikz}[american]
    		
    		% Transformer primary side
    		\draw
    		(0,0) node [transformer core,scale=0.8](T){} % Transformer core
    		(T.A1) node[left]{$N_p$} -- (-2,0) % Primary top winding connection
    		to[sinusoidal voltage source,l_=$80\angle 0^\circ$] (-4,0) -- (-4,-3) node[ground]{};
    		
    		\draw
    		(T.A2) -- (-2,-2.1) -- (-4,-2.1) -- (-4,-2.1);  % Primary bottom winding to ground
    		
    		% Transformer secondary side
    		\draw
    		(T.B1) node[right]{$N_s$} -- (2,0) % Secondary top winding
    		to[R, l_=$120\,\Omega$] (2,-2.1) -- (T.B2); % Resistor and connection to bottom winding
    		
    		% Ground connection for secondary
    		\draw
    		(T.B2) -- (2,-2.1) -- (2,-2.6) -- (2,-3.5) node[ground]{};
    		
    		% Label for transformer
    		\node at (T.base) {K};
    		
    		% Current arrow
    		\draw [->] (-1.5,0.5) -- (-1.5,-0.5) node[midway, right] {$i_s$};
    		
    	\end{circuitikz}
    \end{center}
    \begin{multicols}{4}
    \begin{enumerate}
    \item $0.67\angle 0 \degree$
    \item $2.0\angle 0 \degree$
    \item $2.67\angle 0 \degree$
    \item $10.67\angle 0 \degree$
    \end{enumerate}
	\end{multicols}
    \item In a three-phase, Y-connected squirrel cage induction motor, if $N_s$ is the synchronous speed, $N_r$ is the rotor speed and $s$ is the slip, then the speeds of the airgap field and the rotor field with respect to the stator structure will respectively be
    \hfill{(xe-2007)}
    \begin{multicols}{4}
    \begin{enumerate}
    \item $N_s, sN_r$
    \item $N_s, N_s$
    \item $N_r, N_r$
    \item $N_s, sN_s$
    \end{enumerate}
    \end{multicols}
    \item The equivalent conductance of the forward biased diode, with bias voltage $V$, at the room temperature is
    \hfill{(xe-2007)}
    \begin{enumerate}
    \item constant
    \item proportional to $V$
    \item proportional to $V^{2}$
    \item proportional to $exp\brak{KV}$
    \end{enumerate}
    \item A number is represented as $\vec{\brak{1010\text{ }1010}_2}$ using 8-bits in signed magnitude representation. The decimal number represented is
    \hfill{(xe-2007)}
    \begin{multicols}{4}
    \begin{enumerate}
    \item $-42$
    \item $-85$
    \item $-86$
    \item $-176$
    \end{enumerate}
    \end{multicols}
    \item A 10-bit DAC has a full scale output of $5 V$. The DAC's resolution and step size will respectively be
    \hfill{(xe-2007)}
    \begin{enumerate}
    \item $0.0978\%, 500 mV$
    \item $0.0978\%, 4.88 mV$
    \item $0.195\%, 9.76 mV$
    \item $0.195\%, 500 mV$
    \end{enumerate}
    \item A power source has open circuit voltage of $24V$ and short circuit current of 16 A. At intermediate operating conditions its terminal characteristics is as shown in the figure. The condition under which maximum power can be extracted from the power source is when the
    \hfill{(xe-2007)}
    \begin{minipage}{0.45\textwidth} 
    \begin{circuitikz}[american]
    		
    	% Drawing the box
    	\draw (1,0.8) rectangle (4,3.2);  % Box representing "Power Source"
    	\node at (2,2) {Power Source}; % Label inside the box
    		
    	% Current source in parallel with the box
    	\draw (4,3) to[short] (5,3) 
    	to[I, l_=$I_l$] (5,1) 
    	to[short, -*] (4,1);
    		
    	% Voltage label (V_s) across the box and current source
    	\draw [latex-latex] (3.9,3) -- (3.9,1) node[midway, left] {$V_s$};
    		
   	\end{circuitikz}
	\end{minipage}
	\hfill
	\begin{minipage}{0.45\textwidth}  % Second diagram on the right
		
		% Graph with y-intercept 24V and x-intercept 16A
			\begin{tikzpicture}
				\begin{axis}[
					axis lines=middle,
					xlabel={$I_l$}, ylabel={$V_s$},
					xtick={0, 16}, ytick={0, 24},
					xmin=0, xmax=18, ymin=0, ymax=26,
					grid=none,
					width=5cm,
					height=5cm,
					axis equal
					]
					
					% Plotting the intercept graph
					\addplot[domain=0:16, thick] {24 - (24/16) * x};  % Line equation	
					% Marking the origin
					\node at (axis cs: 0,0) [below left] {O};
					
				\end{axis}
			\end{tikzpicture}
		
	\end{minipage}
    \begin{enumerate}
    \item load current is $16 A$
    \item source voltage is $24 V$
    \item load power is $96 W$
    \item load power is $384 W$
    \end{enumerate}
    \item A $100 kVA, 11 kV/415V$ transformer has $2\%$ winding resistance and $4\%$ leakage reactance. The voltage regulation at rated $kVA, 0.8 pf$ lagging load is
    \hfill{(xe-2007)}
    \begin{enumerate}
    \item $2\%$
    \item $4\%$
    \item $4.8\%$
    \item $6\%$
    \end{enumerate}
    \item The source voltage of the three-phase network shown in the figure is $11 kV$.\\
    \begin{center}
    	\begin{circuitikz}[american]
    		% Delta-connected three-phase source
    		\draw
    		(0,0) to[sinusoidal voltage source] (2,2) % First phase
    		to[sinusoidal voltage source] (4,0) % Second phase
    		to[sinusoidal voltage source, l_=$11\,\mathrm{kV}$] (0,0); % Third phase
    		
    		% Resistor network
    		\draw
    		(2,2) -- (2,3) 
    		to[R=$2\,\Omega$, i=$100\,\mathrm{A}$] (8,3) -- (8,2.5) 
    		to[R] (8,2)
    		to[R] (6,0); 
    		
    		% Remaining resistors and connections
    		\draw
    		(8,2) to[R] (10,0) % Diagonal resistor
    		(4,0) to[R=$2\,\Omega$] (6,0) % Horizontal resistor
    		(10,0) -- (10,-1) % Vertical wire
    		to[R=$2\,\Omega$] (0,-1) % Horizontal bottom resistor
    		(0,0) -- (0,-1); % Closing the bottom connection
    		
    		% Label the source and load
    		\node at (0,-1.5) {Source}; % Source label at the left
    		\node at (10,-1.5) {Load}; % Load label at the right
    		
    	\end{circuitikz}
    \end{center}
    The line voltage at the load end and the phase angle with respect to the source voltage will be
    \hfill{(xe-2007)}
    \begin{enumerate}
    \item $10.7 kV,0\degree$
    \item $10.7 kV,1.08\degree lagging$
    \item $10.7 kV,1.08\degree leading$
    \item $11 kV,1.08\degree lagging$
    \end{enumerate}
    \item A sine-wave voltage at $400 Hz$ feeds the transformer having $50 turns$ in the primary winding as shown in the figure. The transformer core material has a saturation flux density of $1.2 T$ and the hysterisis effects are neglected. The core area is $ 10 cm^{2}$ and its relative permeability is $10^{3}$ till the core reaches saturation.\\
    \begin{center}
   	\begin{circuitikz}[american]
   		
   		% Transformer primary side
   		\draw
   		(0,0) node [transformer core,scale=0.8](T){} % Transformer core
   		(T.A1) node[left]{$N_p$} -- (-2,0.85) % Primary top winding connection
   		to[sinusoidal voltage source,l_=Amplitude A // Frequency 400 Hz] (-2,0) -- (-2,-0.85) --(0,-0.85)
   		;

   		% Transformer secondary side
   		\draw
   		(T.B1) node[right]{$N_s$} -- (2,0.85)
   		(0,-0.85)--(2,-0.85); % Secondary top winding
   		
   		% Label for transformer
   		\node at (T.base) {K};
   		\draw [latex-latex] (2.3,0.85) -- (2.3,-0.85) node[midway, left] {$V_s$};
   		
   	\end{circuitikz}
   \end{center}
    The maximum amplitude of the sine-wave that can be applied on the primary winding without causing saturation under steady state conditions is
    \hfill{(xe-2007)}
    \begin{enumerate}
    \item $24 V$
    \item $48 V$
    \item $75.4 V$
    \item $150.8 V$
    \end{enumerate}
\end{enumerate}
\end{document}


