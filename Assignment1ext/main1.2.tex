%iffalse
\let\negmedspace\undefined
\let\negthickspace\undefined
\documentclass[journal,12pt,twocolumn]{IEEEtran}
\usepackage{cite}
\usepackage{amsmath,amssymb,amsfonts,amsthm}
\usepackage{algorithmic}
\usepackage{graphicx}
\usepackage{textcomp}
\usepackage{xcolor}
\usepackage{txfonts}
\usepackage{listings}
\usepackage{enumitem}
\usepackage{mathtools}
\usepackage{gensymb}
\usepackage{comment}
\usepackage[breaklinks=true]{hyperref}
\usepackage{tkz-euclide} 
\usepackage{listings}
\usepackage{gvv}                                        
%\def\inputGnumericTable{}                                 
\usepackage[latin1]{inputenc}                                
\usepackage{color}                                            
\usepackage{array}                                            
\usepackage{longtable}                                       
\usepackage{calc}                                             
\usepackage{multirow}                                         
\usepackage{hhline}                                           
\usepackage{ifthen}                                           
\usepackage{lscape}
\usepackage{tabularx}
\usepackage{array}
\usepackage{float}


\newtheorem{theorem}{Theorem}[section]
\newtheorem{problem}{Problem}
\newtheorem{proposition}{Proposition}[section]
\newtheorem{lemma}{Lemma}[section]
\newtheorem{corollary}[theorem]{Corollary}
\newtheorem{example}{Example}[section]
\newtheorem{definition}[problem]{Definition}
\newcommand{\BEQA}{\begin{eqnarray}}
\newcommand{\EEQA}{\end{eqnarray}}
\newcommand{\define}{\stackrel{\triangle}{=}}
\theoremstyle{remark}
\newtheorem{rem}{Remark}

% Marks the beginning of the document
\begin{document}
\bibliographystyle{IEEEtran}
\vspace{3cm}

\title{Assignment 1(extras)}
\author{EE24Btech11022 - Eshan sharma}
\maketitle
\newpage
\bigskip

\renewcommand{\thefigure}{\theenumi}
\renewcommand{\thetable}{\theenumi}


\subsection*{D: MCQs with one or more than one answer}
\begin{enumerate}[label=\arabic*.]
    \item There exists a triangle ABC satisfying the conditions
    \hfill{(1986 - 2 mark)}
    \begin{enumerate}[label=(\alph*)]
    \item bsinA $=$ a, A $<\pi/2$
    \item bsinA $>$ a, A $>\pi/2$
    \item bsinA $>$ a, A $<\pi/2$
    \item bsinA $<$ a, A $<\pi/2$, b $>$ a
    \item bsinA $<$ a, A $>\pi/2$, b $=$ a
    \end{enumerate}
    \item In a triangle, the lengths of two larger sides are 10 and 9 respectively. If the angles are in AP, Then length of third side is
    \hfill{(1987 - 2 mark)}
    \begin{enumerate}[label=(\alph*)]
    \item $5-\sqrt{6}$ 
    \item $3\sqrt{3}$
    \item $3$
    \item $5+\sqrt{6}$ 
    \item none
    \end{enumerate}
    \item If in a triangle PQR, $sinP, sinQ, sinR$ are in AP, then
    \hfill{(1998 - 2 mark)}
    \begin{enumerate}[label=(\alph*)]
    \item The altitudes are in AP
    \item The altitudes are in HP
    \item The medians are in GP
    \item The medians are in AP
    \end{enumerate}
    \item Let $A_{0}A_{1}A_{2}A_{3}A_{4}A_{5}$ be a regular hexagon inscribed in a circle of unit radius. Then the product of the lengths of the line segments $A_{0}A_{1}$,$A_{0}A_{2}$ and $A_{0}A_{4}$ is 
    \hfill{(1998 - 2 mark)}
    \begin{enumerate}[label=(\alph*)]
    \item ${\frac{3}{4}}$
    \item $3\sqrt{3}$
    \item 3
    \item ${\frac{3\sqrt{3}}{2}}$
    \end{enumerate}
    \item In $\Delta$ABC, internal angle bisector of $\angle A$ meets side BC in D. DE $\perp$ AD meets AC in E and AB in F. Then
    \hfill{(2006-5M,-1)}
    \begin{enumerate}[label=(\alph*)]
    \item AE is HM of b and c
    \item AD = ${\frac{2bc}{b+c}}cos{\frac{A}{2}}$
    \item EF = ${\frac{4bc}{b+c}}sin{\frac{A}{2}}$
    \item $\Delta$AEF is isosceles
    \end{enumerate}
    \item Let ABC be a triangle such that $\angle ACB = \pi/6$ and let a,b and c denote lengths of the sides opposite to A,B and C respectively. The value(s) of x for which $a = x^{2}+x+1, b = x^{2}-1, c = 2x+1$ is(are)
    \hfill{(2010)}
    \begin{enumerate}[label=(\alph*)]
    \item $-(2+\sqrt{3})$
    \item $1+\sqrt{3}$
    \item $2+\sqrt{3}$
    \item $4\sqrt{3}$
    \end{enumerate}
    \item In a triangle PQR, P is the largest angle and $cosP = \frac{1}{3}$. Further the incircle of the triangle touches the sides PQ,QR and RP at N,L and M respectively, such that the lengths of PN, QL and RM are consecutive even integers. Then possible length(s) of the side(s) of the triangle is(are)
    \hfill{(Jee Adv. 2013)}
    \begin{enumerate}[label=(\alph*)]
    \item 16
    \item 24
    \item 18
    \item 22
    \end{enumerate}
    \item In a triangle XYZ, let x,y,z be the lengths of sides opposite to angles X,Y,Z and $2s = x+y+z$. If ${\frac{s-x}{4}}={\frac{s-y}{3}}={\frac{s-z}{2}}$ and area of the incircle of the triangle XYZ is ${\frac{8\pi}{3}}$
    \hfill{(Jee Adv. 2016)}
    \begin{enumerate}[label=(\alph*)]
    \item area of the triangle is 6$\sqrt{6}$
    \item the radius of circumcirle of XYZ is ${\frac{35\sqrt{6}}{6}}$
    \item $sin\frac{X}{2}sin\frac{Y}{2}sin\frac{Z}{2} = \frac{4}{35}$
    \item $sin^{2}\brak{\frac{X+Y}{2}}$ = $\frac{3}{5}$
    \end{enumerate}
    \item In a triangle PQR, let $\angle PQR = 30\degree$ and the sides PQ and QR have lengths $10\sqrt{3}$ and 10 respectively. Then which of the following statements is(are) TRUE?
    \hfill{(Jee Adv. 2018)}
    \begin{enumerate}[label=(\alph*)]
    \item $\angle QPR = 45\degree$
    \item the area of the triangle PQR is $25\sqrt{3}$ and $\angle QRP = 120\degree$
    \item the radius of the incircle of triangle PQR is $10\sqrt{3}-15$
    \item the radius of circumcirle PQR is $100\pi$
    \end{enumerate}
    \item In a non-right-angle triangle $\Delta PQR$, let p,q,r denote the lengths of the sides opposite to the angles at P,Q,R respectively. The median from R meets the side PQ at S, the perpendicular from P meets the side QR at E, RS and PE
    intersect at O. If p = $\sqrt{3}$, q = 1 and the radius of the circumcircle at $\Delta PQR$ equals 1, then which of the following options is(are)\\ correct.
    \hfill{(Jee Adv. 2018)}
    \begin{enumerate}[label=(\alph*)]
    \item Radius of incircle of $\Delta PQR$ = $\frac{\sqrt{3}}{2}\brak{2-\sqrt{3}}$
    \item Area of $\Delta SOE = \frac{\sqrt{3}}{12}$
    \item Length of OE = $\frac{1}{6}$
    \item Length of RS = $\frac{\sqrt{7}}{2}$
    \end{enumerate}

\end{enumerate}

\subsection*{E:Subjective Problems}

\begin{enumerate}[label=\arabic*.]
    \item A triangle ABC has sides $AB=AC=5 cm$ and $BC =6 cm$. Triangle A'B'C' is the reflection of the triangle ABC in a line parallel to AB placed at a distance of 2 cm from AB, outside the triangle ABC. Triangle A"B"C" is the reflection of the triangle A'B'C' in a line parallel B'C' placed at a distance of 2 cm from B'C' outside the triangle A'B'C'. Find the distance between A and A'.\hfill {(1978)}
    \item 
    \begin{enumerate}[label=(\alph*)]
    \item If a circle is inscribed in a right angled triangle ABC right angled at B, show that the diameter of the circle is equal to AB+BC-AC.
    \item If a triangle is inscribed in a circle, then the product of any two sides of the triangle is equal to the product of the diameter and perpendicular distance of the third side from the opposite vertex. Prove the above statement.
    \end{enumerate}
    \hfill {(1979)}
    \item
    \begin{enumerate}[label=(\alph*)]
    \item A baloon is observed simultaneously from three points A,B and C on a straight road directly beneath it. The angular elevation at B is twice that at A and angular elevation at C is thrice that of A. If the distance between A and B is a and the distance between B and C is b, find height of baloon in terms of a and b.
    \item Find the area of the smaller part of a disc of radius 10 cm, cut off by a chord AB which subtends an angle of 22$\frac{1}{2} \degree$ at the circumference.
    \end{enumerate}
    \hfill {(1980)}
    \item ABC is a triangle. D is the middle point of BC. If AD is perpendicular to AC, then prove that $cosAcosC = \frac{2(c^{2}-a^{2}}{3ac}$.
    \hfill {(1980)}
    \item ABC is a triangle with AB=AC. D is any point on the side BC. E and F are points on the side AB and AC, respectively, such that DE is parallel to AC, and DF is parallel to AB. Prove that \\
    $DF + FA + AE + ED = AB+AC$
    \hfill {(1980)} 
\end{enumerate}
\end{document}
