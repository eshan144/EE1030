%iffalse
\let\negmedspace\undefined
\let\negthickspace\undefined
\documentclass[journal,12pt,onecolumn]{IEEEtran}
\usepackage{cite}
\usepackage{amsmath,amssymb,amsfonts,amsthm}
\usepackage{algorithmic}
\usepackage{graphicx}
\usepackage{textcomp}
\usepackage{xcolor}
\usepackage{txfonts}
\usepackage{listings}
\usepackage{enumitem}
\usepackage{enumitem,multicol}
\usepackage{mathtools}
\usepackage{gensymb}
\usepackage{comment}
\usepackage[breaklinks=true]{hyperref}
\usepackage{tkz-euclide} 
\usepackage{listings}
\usepackage{gvv}                                        
%\def\inputGnumericTable{}                                 
\usepackage[latin1]{inputenc}                                
\usepackage{color}                                            
\usepackage{array}                                            
\usepackage{longtable}                                       
\usepackage{calc}                                             
\usepackage{multirow}                                         
\usepackage{hhline}                                           
\usepackage{ifthen}                                           
\usepackage{lscape}
\usepackage{tabularx}
\usepackage{array}
\usepackage{float}


\newtheorem{theorem}{Theorem}[section]
\newtheorem{problem}{Problem}
\newtheorem{proposition}{Proposition}[section]
\newtheorem{lemma}{Lemma}[section]
\newtheorem{corollary}[theorem]{Corollary}
\newtheorem{example}{Example}[section]
\newtheorem{definition}[problem]{Definition}
\newcommand{\BEQA}{\begin{eqnarray}}
\newcommand{\EEQA}{\end{eqnarray}}
\newcommand{\define}{\stackrel{\triangle}{=}}
\theoremstyle{remark}
\newtheorem{rem}{Remark}

% Marks the beginning of the document
\begin{document}
\bibliographystyle{IEEEtran}
\vspace{3cm}

\title{Vector Algebra}
\author{EE24Btech11022 - Eshan Sharma}
\maketitle

\renewcommand{\thefigure}{\theenumi}
\renewcommand{\thetable}{\theenumi}


\section{MCQs with one correct answer}
\begin{enumerate}
    \item The unit vector which is orthogonal to the vector $3\hat{i} + 2\hat{j} + 6\hat{k}$ and is coplanar with vectors $2\hat{i} + \hat{j} + \hat{k}$ \text{ and } $\hat{i} - \hat{j} + \hat{k}$ is 
    \hfill{(2004S)}
    \begin{multicols}{2}
    \begin{enumerate}
    \item $\frac{2\hat{i} - 6\hat{j} + \hat{k}}{\sqrt{41}}$
    \item $\frac{2\hat{i} - 3\hat{j}}{\sqrt{13}}$
    \item $\frac{3\hat{i} - \hat{k}}{\sqrt{10}}$
    \item $\frac{4\hat{i} + 3\hat{j} - 3\hat{k}}{\sqrt{34}}$
    \end{enumerate}
    \end{multicols}
    \item A variable plane at a distance of one unit from the origin cuts the coordinate axes at $\vec{A}, \vec{B} \text{ and } \vec{C}$. If the centroid $\vec{D} \brak{x,y,z}$ of triangle ABC satisfies the relation $\frac{1}{x^{2}} + \frac{1}{y^{2}} + \frac{1}{z^{2}} = k$, then the value $k$ is
    \hfill{(2005S)}
    \begin{multicols}{2}
    \begin{enumerate}
    \item $3$
    \item $1$
    \item $\frac{1}{3}$
    \item $9$
    \end{enumerate}
    \end{multicols}
    \item If $\vec{a}, \vec{b}, \vec{c}$ are three non-zero, non-coplanar vectors and $\vec{b_1} = \vec{b} - \frac{\vec{b} \cdot \vec{a}}{|\vec{a}^2|} \vec{a}, \vec{b_2} = \vec{b} + \frac{\vec{b} \cdot \vec{a}}{|\vec{a}^2|} \vec{a}, \vec{c_1} = \vec{c} - \frac{\vec{c} \cdot \vec{a}}{|\vec{a}^2|} \vec{a} + \frac{\vec{b} \cdot \vec{c}}{|\vec{c}^2|} \vec{b_1}, \vec{c_2} = \vec{c} - \frac{\vec{c} \cdot \vec{a}}{|\vec{a}^2|} \vec{a} + \frac{\vec{b_1} \cdot \vec{c}}{|\vec{b_1}^2|} \vec{b_1}, \vec{c_3} = \vec{c} - \frac{\vec{c} \cdot \vec{a}}{|\vec{c}^2|} \vec{a} + \frac{\vec{b} \cdot \vec{c}}{|\vec{c}^2|} \vec{b_1}, \vec{c_4} = \vec{c} - \frac{\vec{c} \cdot \vec{a}}{|\vec{c}^2|} \vec{a} = \frac{\vec{b} \cdot \vec{c}}{|\vec{b}^2|} \vec{b_1}$, then the set of orthogonal vectors is 
    \hfill{(2005S)}
    \begin{multicols}{2}
    \begin{enumerate}
    \item $(\vec{a}, \vec{b_1}, \vec{c_3})$
    \item $(\vec{a}, \vec{b_1}, \vec{c_2})$
    \item $(\vec{a}, \vec{b_1}, \vec{c_1})$
    \item $(\vec{a}, \vec{b_2}, \vec{c_2})$
    \end{enumerate}
    \end{multicols}
    \item A plane which is perpendicular to two planes $2x-2y+z=0$ \text{ and } $x-y+2z=4$ passes through $(1,-2,1)$. The distance of the plane from the point $(1,2,2)$ is
    \hfill{(2006 -3M,-1)}
    \begin{multicols}{2} 
    \begin{enumerate}
    \item 0
    \item 1
    \item $\sqrt{2}$
    \item $2\sqrt{2}$
    \end{enumerate}
    \end{multicols}
    \item Let $\vec{a} = \hat{i} + 2\hat{j} + \hat{k}, \vec{b} = \hat{i}-\hat{j}+\hat{k}$ \text{ and } $\vec{c}= \hat{i}+\hat{j}-\hat{k}$. A vector in the plane of $\vec{a}$ \text{ and } $\vec{b}$ whose projection on $\vec{c}$ is $\frac{1}{\sqrt{3}}$, is
    \hfill{(2006-3M,-1)}
    \begin{enumerate}
    \item $4\hat{i} - \hat{j} + 4\hat{k}$
    \item $3\hat{i} + \hat{j} - 3\hat{k}$
    \item $2\hat{i} + \hat{j} - 2\hat{k}$
    \item $4\hat{i} + \hat{j} - 4\hat{k}$
    \end{enumerate}
    \item The number of distinct real values of $\lambda$, for which the vectors $-\lambda^{2}\hat{i} + \hat{j} + \hat{k}$, $\hat{i} - \lambda^{2}\hat{j} + \hat{k}$ \text{ and } $\hat{i} + \hat{j} - \lambda^{2}\hat{k}$ are coplanar, is
    \hfill{(2007 - 3marks)}
    \begin{multicols}{2} 
    \begin{enumerate}
    \item 1
    \item 2
    \item 3
    \item 4
    \end{enumerate}
    \end{multicols}
    \item let $\vec{a},\vec{b},\vec{c}$ be unit vectors such that $\vec{a}+\vec{b}+\vec{c}=\vec{0}$. Which of the following are correct?
    \hfill{(2007- 3marks)}
    \begin{enumerate}
    \item $\vec{a} \times \vec{b} = \vec{b} \times \vec{c} = \vec{c} \times \vec{a} = \vec{0}$
    \item $\vec{a} \times \vec{b} = \vec{b} \times \vec{c} = \vec{c} \times \vec{a} \neq \vec{0}$
    \item $\vec{a} \times \vec{b} = \vec{b} \times \vec{c} = \vec{a} \times \vec{c} \neq \vec{0}$
    \item $\vec{a} \times \vec{b}, \vec{b} \times \vec{c}, \vec{c} \times \vec{a}$ are mutually perpendicular
    \end{enumerate}
    \item The edges of a parallelopiped are of unit length and are parallel to non-coplanar unit vectors $\hat{a},\hat{b},\hat{c}$ such that $\hat{a} \cdot \hat{b}= \hat{b} \cdot \hat{c}= \hat{c} \cdot \hat{a}= \frac{1}{2}$. Then, the volume of the parallelopiped is 
    \hfill{(2008)}
    \begin{multicols}{2}
    \begin{enumerate}
    \item $\frac{1}{\sqrt{2}}$
    \item $\frac{1}{2\sqrt{2}}$
    \item $\frac{\sqrt{3}}{2}$
    \item $\frac{1}{\sqrt{3}}$
    \end{enumerate}
    \end{multicols}
    \item Let two non-collinear unit vectors $\hat{a}$ and $\hat{b}$ form an acute angle. A point $\vec{P}$ moves so that at any time t the position vector $\overrightarrow{OP}$ (where O is the origin) is given by $\hat{a}\cos{t} + \hat{b}\sin{t}$. When $\vec{P}$ is farthest from origin $\vec{O}$, let $\vec{M}$ be the length of $\overrightarrow{OP}$ and $\hat{u}$ be the unit vector along $\overrightarrow{OP}$. Then,
    \hfill{(2008)}
    \begin{enumerate}
    \item $\hat{u} = \frac{\hat{a}+\hat{b}}{|\hat{a}+\hat{b}|}$  and $M = (1+\hat{a} \cdot \hat{b})^{1/2}$
    \item $\hat{u} = \frac{\hat{a}-\hat{b}}{|\hat{a}-\hat{b}|}$  and $M = (1+\hat{a} \cdot \hat{b})^{1/2}$
    \item $\hat{u} = \frac{\hat{a}+\hat{b}}{|\hat{a}+\hat{b}|}$  and $M = (1+2\hat{a} \cdot \hat{b})^{1/2}$
    \item $\hat{u} = \frac{\hat{a}-\hat{b}}{|\hat{a}-\hat{b}|}$  and $M = (1+2\hat{a} \cdot \hat{b})^{1/2}$
    \end{enumerate}
    \item Let $\vec{P}(3,2,6)$ be a point in space and $\vec{Q}$ be a point on the line \\ 
    $\vec{r} = (\hat{i} - \hat{j} + 2\hat{k}) + \mu(-3\hat{i} +\hat{j}+5\hat{k}.$
    \\ Then the value of $\mu$ for which the vector $\overrightarrow{PQ}$ is parallel to the plane $x-4y+3z=1$ is
    \hfill{(2009)}
    \begin{multicols}{2}
    \begin{enumerate}
    \item $\frac{1}{4}$
    \item $-\frac{1}{4}$
    \item $\frac{1}{8}$
    \item $-\frac{1}{8}$
    \end{enumerate}
    \end{multicols}
    \item If $\vec{a}, \vec{b}, \vec{c},$ and $\vec{d}$ are unit vectors such that $(\vec{a} \times \vec{b}) \cdot (\vec{c} \times \vec{d}) = 1$ and $\vec{a} \cdot \vec{c} = \frac{1}{2}$, then
    \hfill{(2009)}
    \begin{enumerate}
    \item $\vec{a}, \vec{b}, \vec{c}$ are non-coplanar
    \item $\vec{b}, \vec{c}, \vec{d}$ are non-coplanar
    \item $\vec{b}, \vec{d}$ are non-parallel
    \item $\vec{a}, \vec{d}$ are parallel and $\vec{b}, \vec{c}$ are parallel 
    \end{enumerate}
    \item A line with positive direction cosines passes through the point $\vec{P}(2,-1,2)$ and makes equal angles with the coordinate axes. The line meets the plane $2x+y+z=9$ at point $\vec{Q}$. The length of the line segment PQ equals 
    \hfill{(2009)}
    \begin{multicols}{2}
    \begin{enumerate}
    \item 1
    \item $\sqrt{2}$
    \item $\sqrt{3}$
    \item 2
    \end{enumerate}
    \end{multicols}
    \item Let $\vec{P}, \vec{Q}, \vec{R} and \vec{S}$ be the points on the plane with position vectors $-2\hat{i} -\hat{j},4\hat{i},3\hat{i}+3\hat{j} and -3\hat{i}+2\hat{j}$ respectively. The quadrilateral PQRS must be a 
    \hfill{(2010)}
    \begin{enumerate}
    \item parallelogram, which is neither a rhombus nor a rectangle 
    \item square 
    \item rectangle, but not a square
    \item rhombus, but not a square 
    \end{enumerate}
    \item Equation of the plane containing the straight line $\frac{x}{2}=\frac{y}{3}=\frac{z}{4}$ and perpendicular to the plane containing the straight lines $\frac{x}{3}=\frac{y}{4}=\frac{z}{2}$ and $\frac{x}{4}=\frac{y}{2}=\frac{z}{3}$ is 
    \hfill{(2010)}
    \begin{enumerate}
    \item $x+2y-2z=0$
    \item $3x+2y-2z=0$
    \item $x-2y+z=0$
    \item $5x+2y-4z=0$
    \end{enumerate}
    \item If the distance of the point $\vec{P}(1,-2,1)$ from the plane $x+2y-2z=\alpha$, where $\alpha>0$, is 5, then the foot of the perpendicular from $\vec{P}$ to the plane is
    \hfill{(2010)}
    \begin{enumerate}
    \item $\brak{\frac{8}{3}, \frac{4}{3}, \frac{-7}{3}}$
    \item $\brak{\frac{4}{3}, \frac{-4}{3}, \frac{1}{3}}$
    \item $\brak{\frac{1}{3}, \frac{2}{3}, \frac{10}{3}}$
    \item $\brak{\frac{2}{3}, \frac{-1}{3}, \frac{5}{2}}$
    \end{enumerate}
\end{enumerate}
\end{document}


